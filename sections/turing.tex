%!TEX root = slides.tex

\section{Turing machines}


\begin{frame}{Visualisation}
  \begin{adjustbox}{max width={0.9\textwidth},center} 
  \begin{tikzpicture}
  \tikzstyle{every path}=[very thick]
  
  \edef\sizetape{0.7cm}
  \tikzstyle{tmtape}=[draw,minimum size=\sizetape]
  \tikzstyle{tmhead}=[arrow box,draw,minimum size=.8cm,arrow box arrows={east:.25cm, west:0.25cm}]
  
  \begin{scope}[start chain=1 going right,node distance=-0.15mm]
  \node [on chain=1,tmtape] {1};
  \node [on chain=1,tmtape] (input) {0};
  \node [on chain=1,tmtape] {1};
  \node [on chain=1,tmtape] {1};
  \node [on chain=1,tmtape] {0};
  \node [on chain=1,tmtape] {1};
  \node [on chain=1,tmtape] {0};
  \node [on chain=1,tmtape] {};
  \node [on chain=1,tmtape,draw=none] {$\ldots$};
  \end{scope}

  \node [tmhead,yshift=.7cm] at (input.north) (head) {$q_1$};
  \end{tikzpicture}
  \end{adjustbox}

  \vspace{1.5cm}

  \begin{adjustbox}{max width={0.9\textwidth},center} 
    \begin{tikzpicture}
    \tikzstyle{every path}=[very thick]
    
    \edef\sizetape{0.7cm}
    \tikzstyle{tmtape}=[draw,minimum size=\sizetape]
    \tikzstyle{tmhead}=[arrow box,draw,minimum size=.8cm,arrow box arrows={east:.25cm, west:0.25cm}]
    
    \begin{scope}[start chain=1 going right,node distance=-0.15mm]
    \node [on chain=1,tmtape] {1};
    \node [on chain=1,tmtape] {0};
    \node [on chain=1,tmtape] (input) {1};
    \node [on chain=1,tmtape] {1};
    \node [on chain=1,tmtape] {0};
    \node [on chain=1,tmtape] {1};
    \node [on chain=1,tmtape] {0};
    \node [on chain=1,tmtape] {};
    \node [on chain=1,tmtape,draw=none] {$\ldots$};
    \end{scope}
    
    \node [tmhead,yshift=.7cm] at (input.north) (head) {$q_2$};
    \end{tikzpicture}
  \end{adjustbox}
\end{frame}

\begin{frame}{State Table}
  \begin{table}
    \centering
    \begin{tabular}{cc|ccc}
    \toprule
        State & Input & Write & Move & Next \\
    \midrule
        $q_0$ & $\sqcup$ & $\sqcup$ & L & $q_a$ \\
        $q_0$ & 0 & 0 & R & $q_0$ \\
        $q_0$ & 1 & 1 & R & $q_1$ \\
    \midrule
        $q_1$ & $\sqcup$ & $\sqcup$ & L & $q_f$ \\
        $q_1$ & 0 & 0 & R & $q_1$ \\
        $q_1$ & 1 & 1 & R & $q_0$ \\
    \bottomrule
    \end{tabular}
  \end{table}
  

  \[ \delta(q_i, \gamma_n) \rightarrow (q_j, \gamma_m, L/R) \]
\end{frame}



\begin{frame}{Notation}
\begin{description}
  \item[$Q$] Set of states (finite).
  \item[$\Sigma$] Input alphabet, subset of $\Gamma \setminus \{ \sqcup \} $.
  \item[$\Gamma$] Tape alphabet (finite).
  \item[$\sqcup$] Blank symbol, element of $\Gamma$.
  \item[$\delta$] Transition function, $\delta: Q \times \Gamma \rightarrow Q \times \Gamma \times \{L,R\}$.
  \item[$q_0$] Start state, $\in Q$.
  \item[$q_a$] Accept state, $\in Q$.
  \item[$q_r$] Reject state, $\in Q$, $\neq q_a$.
  \vspace{0.3cm}
  \item[$M$] Turing Machine: $(Q, \Sigma, \Gamma, \delta , q_0, q_a, q_r )$.
\end{description}
\end{frame}


\begin{frame}{Blank symbol}

\begin{description}
  \item[Blank] is generally the only difference between the tape alphabet and the input alphabet.
  \item[Definitions] of Turing machines generally don't disallow other symbols in the difference, but there's always at least a blank.
  \item[Empty cells] of the tape in the machine are said to contain the blank symbol.
  \item[Importantly] the blank symbol marks the end of the input on the tape.
  \item[That's why] the input cannot contain the blank symbol.
\end{description}

\end{frame}


\begin{frame}{Sets and alphabets}

\begin{description}
  \item[Recall] sets are just collections of objects called elements.
  \item[Sets] have two important attributes -- an object is either in the set or not, and all elements are distinct.
  \item[Alphabets] in the definition of Turing machines are just sets, and their elements are called symbols.
  \item[Strings] are finite sequences (i.e. ordered lists) of symbols over alphabets.
  \item[$\epsilon$] is the empty string.
  \item[$\mathbb{A}^*$] is the set containing all of the strings over the alphabet A, including the empty string.
  \item[$|w|$] denotes the length of a string $w$, e.g. if $w = xxyzxy$ then $|w| = 6$.
\end{description}
\end{frame}

\begin{frame}{Alphabets and strings}
\metroset{block=fill}
  \begin{alertblock}{Examples}
    The following are examples of strings over the alphabet $\{0,1\}$:
    \begin{itemize}
      \item 100110
      \item 111
      \item 0
      \item $\epsilon$
    \end{itemize}
  \end{alertblock}
\metroset{block=transparent}
\begin{alertblock}{Single character strings}
Note the distinction between a symbol in an alphabet and the string containing a single string.
They look the same, but one is a symbol and one is a string. 
This is akin to the distinction in C between the character 'a' and the string literal "a".
\end{alertblock}
\end{frame}


\begin{frame}{Languages}

\begin{description}
  \item[Language] is a set of strings.
  \item[Turing machines] accept some strings as inputs.
  \item[Accepted] language of a Turing machine is the set of strings it accepts.
  \item[Halting] -- given a string, a Turing machine will either accept it, reject it, or never stop (fail to halt).
  \item[Decide] -- a Turing machine that halts on all inputs is called a decider for the language it accepts.
  \item[Turing-decidable] -- a language that some Turing machine decides.
\end{description}

\end{frame}



\begin{frame}{Turing's Second Example}
  \begin{quote}
    We can construct a machine to compute the sequence
    \[ 001011011101111011111 \ldots \]
    The machine is to be capable of five $m$-configurations, viz. $\mathfrak{o}$, $\mathfrak{q}$, $\mathfrak{p}$, $\mathfrak{f}$, $\mathfrak{b}$ and of printing $a$, $x$, $0$, $1$.
    The first three symbols on the tape will be $aa0$; the other figures follow on alternate squares.
    On the intermediate squares we never print anything but $x$.
    These letters serve to keep the place for us and are erased when we have finished with them.
    We also arrange that in the sequence of figures on alternate squares there shall be no blanks.
  \end{quote}
\end{frame}



\begin{frame}{Turing's Second Example: Table}
  \begin{table}
  \resizebox{\textwidth}{!}{
    \centering
    \begin{tabular}{cccc}
      \emph{m-config.}  & \emph{symbol}  & \emph{operations} & \emph{final m-config.} \\
      $\mathfrak{b}$ & & $Pe,R,Pe,R,P0,R,R,P0,L,L$  & $\mathfrak{o}$ \\
      & & & \\
      \multirow{2}{*}{$\mathfrak{o}$} & $1$ & $R,Px,L,L,L$ & $\mathfrak{o}$ \\
      & 0 & & $\mathfrak{q}$ \\
      & & & \\
      \multirow{2}{*}{$\mathfrak{q}$} & Any (0 or 1) & $R,R$ & $\mathfrak{q}$ \\
      & None & $P1,L$ & $\mathfrak{p}$ \\
      & & & \\
      \multirow{3}{*}{$\mathfrak{p}$} & x & $E,R$ & $\mathfrak{q}$ \\
      & e & R & $\mathfrak{f}$ \\
      & None & $L,L$ & $\mathfrak{p}$ \\
      & & & \\
      \multirow{2}{*}{$\mathfrak{f}$} & Any & $R,R$ & $\mathfrak{f}$ \\
      & None & $P0,L,L$ & $\mathfrak{o}$ \\
    \end{tabular}}
  \end{table}
\end{frame}


\begin{frame}[fragile]{Turing's Second Example: JavaScript 1}
\begin{minted}{javascript}
// The contents of the tape.
var tape = []
s// The current position of the machine on the tape.
var pos = 0
// The current state;
var state = b;
\end{minted}
\end{frame}

\begin{frame}[fragile]{Turing's Second Example: JavaScript 2}
\begin{minted}{javascript}
// Writes a symbol to the current cell on the tape.
function write(sym) {
  tape[pos] = sym;
}

// Returns true iff the current cell contains sym.
function read(sym) {
  return sym == tape[pos] ? true : false;
}
\end{minted}
\end{frame}

\begin{frame}[fragile]{Turing's Second Example: JavaScript 3}
\begin{minted}{javascript}
// Erases the symbol in the current cell of the tape.
function erase() {
  delete tape[pos];
}

// Returns true iff the current cell is blank.
// Returns true iff the current cell is blank.
function blank() {
  return typeof(tape[pos])
    == 'undefined' ? true : false;
}
\end{minted}
\end{frame}

\begin{frame}[fragile]{Turing's Second Example: JavaScript 4}
\begin{minted}{javascript}
function b() {
  write('e');
  pos++;
  write('e');
  pos++;
  write('0');
  pos++;
  pos++;
  write('0');
  pos--;
  pos--;
  state = o;
}
\end{minted}
\end{frame}
