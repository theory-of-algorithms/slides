%!TEX root = slides.tex

\title{Theory of Algorithms}
\subtitle{}
\author{ian.mcloughlin@gmit.ie}
\date{}


\begin{frame}
	\titlepage
\end{frame}

\begin{frame}
	\frametitle{Topics}
	\tableofcontents
\end{frame}

\section{Python}


\begin{frame}{About Python}
  \begin{description}
    \item[January 1994] -- Python 1.0.0 released.
    \vspace{0.25cm}
    \item[Guido van Rossum] -- Designer/Author of Python.
    \vspace{0.25cm}
    \item[Current versions] -- 3.5.1 and 2.7.11.
    \vspace{0.25cm}
    \item[Interpreted] -- Python implementation must be present at runtime.
    \vspace{0.25cm}
    \item[Off-side rule] -- Blocks identified by indentation, as opposed to curly braces.
    \vspace{0.25cm} 
    \item[Popularity] -- IEEE Spectrum ranks it as the fourth most popular language (July 2015).
    \vspace{0.25cm}
    \item[Community] -- Python Enhancement Proposals, notably \href{https://www.python.org/dev/peps/pep-0008/}{PEP 8: The Python Style Guide}.
    \vspace{0.25cm} 
  \end{description}
  \citeurl{spectrum.ieee.org/computing/software/the-2015-top-ten-programming-languages}
\end{frame}


\begin{frame}{Guido van Rossum}
  \begin{columns}
    \begin{column}{0.2\textwidth}
      \includegraphics[height=2in]{img/guido_van_rossum.jpg}
    \end{column}
    \begin{column}{0.6\textwidth}
      \begin{itemize}
    		\item Started Python as a hobby.
        \vspace{0.25cm}
    		\item Worked for Google, half-time spent on Python.
        \vspace{0.25cm}
    		\item Now works at Dropbox.
        \vspace{0.25cm}
        \item Benevolent dictator for life (BDFL).
      \end{itemize}
    \end{column}
  \end{columns}
\end{frame}

\begin{frame}[fragile]{Conditions}
  \begin{minted}[linenos, frame=lines, framesep=2mm]{python}
x = int(raw_input("Please enter an integer: "))
if x < 0:
  x = 0
  print 'Negative changed to zero'
elif x == 0:
  print 'Zero'
elif x == 1:
  print 'Single'
else:
  print 'More'
  \end{minted}
  \citeurl{docs.python.org/2/tutorial}
\end{frame}

\begin{frame}[fragile]{Loops}
  \begin{minted}[linenos, frame=lines, framesep=2mm]{python}
# A for loop.
a = ['Mary', 'had', 'a', 'little', 'lamb']
for i in range(len(a)):
  print(i, a[i])
  \end{minted}
  \begin{minted}[linenos, frame=lines, framesep=2mm]{python}
# A while loop.
a, b = 0, 1
while b < 1000:
  print(b)
  a, b = b, a+b
  \end{minted}
  \citeurl{docs.python.org/3/tutorial}
\end{frame}

\begin{frame}[fragile]{Functions}
  \begin{minted}[linenos, frame=lines, framesep=2mm]{python}
# write Fibonacci series up to n
def fib(n):   
  """Print a Fibonacci series up to n."""
  a, b = 0, 1
  while a < n:
    print(a)
    a, b = b, a+b
  \end{minted}
  \citeurl{docs.python.org/3/tutorial}
\end{frame}

\begin{frame}{CPython}
  \begin{description}
    \item[Reference implementation] -- Many different Python implementations exist.
    \vspace{0.25cm}
    \item[Version 3] -- Broke backwards compatibility (somewhat).
    \vspace{0.25cm}
    \item[Unladen Swallow] -- Google attempt to fix some Python problems.
    \vspace{0.25cm}
    \item[Modules] -- Lots of great Python modules available.
  \end{description}
  \citeurl{www.python.org}
\end{frame}

\begin{frame}[fragile]{Lists}
	\begin{description}
		\item[Lists] in Python are usually written as comma-separated values between square brackets.
		\item[Types] -- elements of a list don't have to have the same types.
		\item[Slicing] is possible, where we take a sublist of the list.
		\item[Assignment] to slices is possible.
		\item[len()] is a built-in function that returns the length of a list.
		\item[range()] is a built-in function that returns a list of numbers. Note: it returns an \emph{iterator}.
	\end{description}
	\begin{minted}[linenos, frame=lines, framesep=2mm]{python}
letters = ['a', 'b', 'c']
letters[1:] = ['c', 'd']
range(10) # [0,1,2,3,4,5,6,7,8,9]
  \end{minted}
  \citeurl{docs.python.org/3/tutorial}
\end{frame}

\begin{frame}[fragile]{Strings}
	\begin{description}
	  \item[Strings] are a lot like lists in Python.
	  \item[Assignment] to slices is not allowed, however.
	\end{description}
	\begin{minted}[linenos, frame=lines, framesep=2mm]{python}
words = "This is a sentence."
words[8]         # a
words[5:7]       # is
words[:7]        # This is
words[10:]       # sentence.
words[17:9:-1]   # ecnetnes

len(words)       # 19
"One" + "Two"    # OneTwo
  \end{minted}
	\citeurl{docs.python.org/3/tutorial}
\end{frame}

\begin{frame}[fragile]{Functions}
	\begin{description}
	  \item[def] is the keyword for defining a function.
	  \item[Parameters] can be given defaults, so that they are optional.
	\end{description}
	\begin{minted}[linenos, frame=lines, framesep=2mm]{python}
def axn(x, a=1, n=2):
	return a*(x**n)    # ax^n
 
axn(3)       # 9
axn(3, 2)    # 18
axn(3, 2, 3) # 54
axn(3, n=3)  # 27
  \end{minted}
	\citeurl{docs.python.org/3/tutorial}
\end{frame}

\begin{frame}[fragile]{List comprehensions}
	\begin{description}
	  \item[Comprehensions] are quick ways of creating lists from other lists.
	\end{description}
	\begin{minted}[linenos, frame=lines, framesep=2mm]{python}
nos = range(5) # [0, 1, 2, 3, 4]
squares = [i*i for i in nos] # [0, 1, 4, 9, 16]
oddsqs = [i*i for i in nos if i % 2 == 1] # [1, 9]
  \end{minted}
	\citeurl{docs.python.org/3/tutorial}
\end{frame}



\begin{frame}[fragile]{map()}
	\begin{description}
	  \item[map()] takes a function and a list.
	  \item[New list] -- it returns a new generator, which is the original list with the function applied to each element.
	\end{description}
	\begin{minted}[linenos, frame=lines, framesep=2mm]{python}
map(len, words)
list(map(len, words))
  \end{minted}
	\citeurl{docs.python.org/3/tutorial}
\end{frame}


\begin{frame}[fragile]{Lambda functions}
	\begin{description}
	  \item[lambda] functions are short, inline functions.
	  \item[Nameless] -- lambda functions need not have a name.
	\end{description}
	\begin{minted}[linenos, frame=lines, framesep=2mm]{python}
lambda x: x + n
  \end{minted}
	\citeurl{docs.python.org/3/tutorial}
\end{frame}

\section{Timing Algorithms}

\section{Functional Programming}

\section{Turing Machines}

\section{Complexity Classes}