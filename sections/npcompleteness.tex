%!TEX root = slides.tex

\section{NP completeness}


\begin{frame}{Non-deterministic polynomial time}
  \begin{definition}
  A problem is in the $NP$ complexity class if it is solvable by a non-deterministic Turing Machine in polynomial time. A non-deterministic Turing Machine is one which may not have a unique action to take for some or all states and inputs.
  \end{definition}
  
  \vspace{0.5cm}
  \metroset{block=fill}
  \begin{block}{$P$ is a subset of $NP$}
  The $P$ complexity class is a subset of $P$ because all polynomial time solvabled problems can be modelled using Nondeterministic Turing Machinses.
  \end{block}

  \citeurl{mathworld.wolfram.com/P-Problem.html}
\end{frame}


\begin{frame}{$NP$-complete problems}
  \begin{definition}
    An problem is $NP$-hard if each problem in $NP$ can be reduced to it in polynomial time.
  \end{definition}
  
  \vspace{0.5cm}
  {\metroset{block=fill}
  \begin{block}{$NP$-hard problems are hard}
  $NP$-hard problems are at least as hard to solve as the hardest $NP$ problems.
  Note that $NP$-hard problems don't have to be $NP$.
  \end{block}}
  
  \vspace{0.5cm}
  \begin{definition}
    An problem is $NP$-complete if it's in $NP$ and in $NP$-hard.
  \end{definition}
\end{frame}


\begin{frame}{Subset sum problem}
   
   {\metroset{block=fill}
  \begin{block}{Problem}
    Given a set of integers $S$, is there a non-empty subset whose elements sum to zero?
  \end{block}}

  \vspace{0.5cm}
  
  \begin{block}{Example}
    Does $\{ 1, 3, 7, -5, -13, 2, 9, -8 \}$ have such a subset?
  \end{block}
  
  \vspace{0.5cm}

  \begin{block}{Note}
     If somebody suggests a soltuion, it is very quick to check it.
    Being able to quickly verify a solution is a characteristic of $NP$ problems.
  \end{block}
\end{frame}


\begin{frame}{Propositional logic}
  \begin{description}
    \item[Literals] are Boolean variables (can be True or False), and their negations. They are represented by lower case letters like $a$ and $x_i$.
    \vspace{0.3cm}
    \item[Clause] are expressions based on literals, that evaluate as True or False based on the literals. We use not, and and or on the literals. We'll sometimes call them expressions.
    \vspace{0.3cm}
    \item[Not] is depicted by $\neg$. So ``not $a$'' is denoted by $\neg a$.
    \vspace{0.3cm}
    \item[Or] is depicted by $\vee$. So ``$a$ or $b$'' is denoted by $a \vee b$.
    \vspace{0.3cm}
    \item[And] is depicted by $\wedge$. So ``$a$ and $b$'' is denoted by $a \wedge b$.
  \end{description}
\end{frame}


\begin{frame}{Normal forms}
  \begin{description}
    \item[CNF] A clause is in Conjunctive Normal Form if it is a ``conjunction of disjunctions'': $(a \vee b) \wedge (\neg a \vee c) \wedge d$.
    \vspace{0.5cm}
    \item[DNF] A cluse is in Disjunctive Normal Form if it is a ``disjunction of conjunctions'': $(a \wedge b) \vee (\neg a \wedge c) \vee d$.
  \end{description}
  \vspace{0.5cm}
  {\metroset{block=fill}\begin{block}{Converting to CNF and DNF}
    Every Boolean expression can be converted to CNF, and every Boolean expression can also be converted to DNF.
  \end{block}}
\end{frame}


\begin{frame}{Four laws}
  The following four laws can be used to convert expressions to CNF and DNF.
  The first two are known as De Morgan's laws, and the latter two are called the distributivity laws.
  \vspace{0.5cm}
  {\metroset{block=fill}\begin{block}{Conversion laws}
    \[ \neg ( a \vee b) = \neg a \wedge \neg b \]
    \[ \neg ( a \wedge b) = \neg a \vee \neg b \]
    \[ c \wedge ( a \vee b) = (c \wedge a) \vee (c \wedge b) \]
    \[ c \vee ( a \wedge b) = (c \vee a) \wedge (c \vee b) \]
  \end{block}}
\end{frame}


\begin{frame}{Boolean Satisfiability Problem (SAT)}
  {\metroset{block=fill}
    \begin{block}{The problem}
    We're often interested in knowing if there is any setting of the variables in a Boolean expression that makes the expression true.
    Another way of asking the question is: is the expression satisfiable?
    The real question is, is there a polynomial time algorithm that takes as input \emph{any} Boolean expression and outputs true if there is any setting, and false otherwise.
  \end{block}}
  \vspace{0.5cm}
  The problem is the prototypical NP-complete problem.
  Note that it's quick to check the correctness of a solution for a given expression. 
\end{frame}


\begin{frame}{CNF SAT}
  
  {\metroset{block=fill}
    \begin{block}{Only using CNF}
    All Boolean expressions can be converted in CNF.
    However, it's not always possible to do that in polynomial time.
    We can though, in polynomial time, create new expressions using some extra(neous) variables that are satisfiable if and only if the original expression is.
  \end{block}}
  \vspace{0.5cm} 
\end{frame}


\begin{frame}{$k$-SAT}
  
  {\metroset{block=fill}
    \begin{block}{$k$-SAT}
     $k$-SAT is like SAT except that all expressions must be in CNF and each clause must be a disjunction of $k$ literals.
  \end{block}}
  \vspace{0.25cm}
  \begin{description}
  \item[2-SAT] $(a \vee b) \wedge (\neg c \vee d) \wedge \ldots$
  \vspace{0.25cm}
  \item[3-SAT] $(a \vee b \vee c) \wedge (\neg c \vee d \vee \neg a) \wedge \ldots$
  \end{description}
    \vspace{0.25cm}
   \begin{description}
     \item[2-SAT] is not NP-complete. There are polynomial time algorithms that solve it.
     \vspace{0.25cm}
     \item[3-SAT] is NP complete.
   \end{description}
\end{frame}


\begin{frame}{3-SAT is NP-Complete}

  {\metroset{block=fill}
    \begin{block}{Reduction}
    3-SAT is a special case of SAT, so 3-SAT must be in NP.
    We can reduce SAT to 3-SAT in polynomial time.
    First take the expression and convert it to a CNF expression (in polynomial time).
    Then we just need to convert each clause into a CNF expression with 3 literals per clause.
    
    Suppose we have a clause with 1 literal: $a$.
    Convert this to $(a \vee u_1 \vee u_2) \wedge (a \vee u_1 \vee \neg u_2) \wedge (a \vee \neg u_1 \vee u_2) \wedge (a \vee \neg u_1 \vee \neg u_2)$.
    Suppose we have a clause with 2 literals: $a \vee b$.
    Convert this to $(a \vee b \vee u_1) \wedge (a \vee b \vee \neg u_1))$.
    
    Now suppose we have a clause with $n$ literals: $a \vee b \vee c \vee \ldots$.
    Convert this to $(a \vee b \vee u_1) \wedge (c \vee \neg u_1 \vee u_2)) \wedge \ldots \wedge (i \vee \neg u_{n-4} \vee u_{n-3}) \wedge (j \vee k \vee \neg u_{n-3})$.
  \end{block}}

\end{frame}